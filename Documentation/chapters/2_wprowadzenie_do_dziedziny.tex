\chapter{Wprowadzenie do dziedziny (Paweł Januszewski)}
\label{chap:field}

Escape Room ma wiele zastosowań, w zależności od intencji osób biorących w nim udział, może on być dostosowany do celów rekreacyjnych lub edukacyjnych. Pokój Zagadek często zawiera określony temat i historię wokół której zbudowane są zagadki w pokoju oraz ogólny postęp całego doświadczenia. 
Zagadki mogą być skonstruowane dla jednego uczestnika lub grupy, w celu kooperacji między uczestnikami. Celem Escape Room jest rozwiązanie wszystkich zagadek w odpowiedniej kolejności i ucieczka z pokoju, ucieczka oznacza osiągnięcie końca zabawy.

W wersji rekreacyjnej, Pokój Zagadek może być zbudowany w celu rozrywki grupowej, często grupy znajomych. Współpraca w rozwiązywaniu zagadek oraz radość ze wspólnego sukcesu jest często powodem do odwiedzenia tego rodzaju rozrywek. 

W wersji edukacyjnej, Pokój Zagadek jest bardziej wyspecjalizowany, historia jest mniej istotna w tym przypadku, zagadki oraz ich mechanika stanowią główny cel pokoju.
Celem jest nauka poprzez zabawę, przykładem może być pokój o tematyce matematycznej gdzie zagadki wymagają znajomości matematyki w celu ich rozwiązania, rozwiązanie zagadki utrwala lub przyswaja daną wiedzę dla gracza.


Escape Room wymaga przygotowania pokoju oraz zagadek przed rozpoczęciem gry, wkład pracy przy przygotowaniach jest też obecny podczas kolejnego podejścia do zabawy, przez tą samą lub inną grupę osób. W celu rozwiązania tego problemu można użyć wirtualnej przestrzeni gry, daje to możliwość zmiany elementów pokoju oraz odświeżenia zagadek z minimalnym wkładem pracy.
Jaskinia Zanurzenia Wirtualnego CAVE (Cave automatic virtual environment) stała się popularnym sposobem uczestnictwa Pokoju Zagadek. W przeciwieństwie do tradycyjnego fizycznego pokoju, gracz może przemieszczać się między wieloma pokojami bez zmiany swojej lokalizacji w CAVE. Dzięki zmianie poziomów jesteśmy w stanie wyspecjalizować zagadki do poszczególnych pokoi, tworząc łańcuch poziomów o określonym rodzaju wyzwań.
CAVE jest pomieszczeniem którego ściany są ekranami do projekcji stereoskopowej, zawiera od 3 do 6 ekranów. Do poruszania się oraz interakcji z elementami pokoju używamy specjalnego kontrolera zwanego różdżką, można przypisać funkcje do guzików dostępnych na kontrolerze w celu lepszej kontroli lub ułatwienia zabawy, na przykład obracanie pokoju.
Dzięki wirtualnemu pokojowi możemy w szybki sposób wymienić rodzaj tematyki w którym gracz bierze udział, dostosowanie pokoju do potrzeb naukowych osoby biorącej udział w rozwiązywaniu zagadek stanowi przewagę nad tradycyjnymi Pokojami Zagadek.

Escape Room o tematyce fizycznej jest pokojem skupiającym się na rozwiązywaniu zagadek z różnych dziedzin fizyki, każdy pokój zawiera wyzwania z konkretnego działu fizyki, są one ułożone w sposób wymagający wiedzy o danej tematyce. 



\section{Podrozdział}
