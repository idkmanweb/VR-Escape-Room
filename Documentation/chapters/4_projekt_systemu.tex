\chapter{Projekt systemu (Piotr Chojnowski)}
\label{chap:project}

\section{Ogólna struktura aplikacji}
Aplikacja ma postać 11 pokoi w rozmiarze 3.4m x 3.4m x 3.4m. Każdy pokój zawiera rozbudowaną zagadkę z jednej z 11 dziedzin fizyki na poziomie szkoły średniej: mechanika, mechanika bryły sztywnej, grawitacja i elementy astronomii, drgania, termodynamika, elektrostatyka, prąd elektryczny, magnetyzm, fale i optyka, fizyka atomowa, elementy fizyki relatywistycznej i fizyka jądrowa.

Gracz zaczyna w pierwszym pokoju. Po rozwiązaniu w nim zagadki przechodzi do kolejnego, wygrywa gdy ukończy ostatni pokój. Wszystkie zagadki łączy ciągła fabuła.

\section{Projekty zagadek}
Zagadka z fali i optyki polega na odpowiednim ustawieniu poprawnych soczewek w lukach w labiryncie na ścianie, tak aby wiązka światła dotarła do jego końca. Przy drugiej ścianie znajduje się stół z różnymi opisanymi soczewkami, które gracz może wybrać. Gracz używa kontrolera aby przenosić soczewki, które automatycznie wchodzą do luki gdy są wysarczająco blisko niej. Zagadka liczy się jako ukończona gdy wiązka światła przejdzie przez labirynt.

Zagadka z prądu elektrycznego polega na uzupełnieniu układu elektrycznego aby dostarczał wystarczające napięcie prądu jednocześnie nie spalając układu. Gracz dostosowuje parametry elementów takich jak rezystory, zasilanie i połączenia przewodników. Zagadka liczy się jako ukończona gdy gracz ukończy dwa układy - pierwszy służy jako poradnik wyjaśniający sterowanie.

Zagadka z elektrostatyki stawia gracza w różnych polach elektrostatycznych, gdzie gracz musi dobrać ładunki tak, aby ustabilizować ruch cząsteczek w tych polach. Zagadka liczy się jako ukończona po ustabilizowaniu 3 pól ze zwiększającym się poziomem skomplikowania.
