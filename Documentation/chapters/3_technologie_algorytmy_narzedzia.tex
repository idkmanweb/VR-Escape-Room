\chapter{Technologie, algorytmy i narzędzia (Piotr Chojnowski)}
\label{chap:algs}

\section{Technologie}
Aplikacja została stworzona na system CAVE. System ten na postać pokoju w którym ściany zastąpione są ekranami do projekcji stereoskopowej. Został on wynaleziony przez Carolinę Cruz-Neira, Daniela J. Sandin i Thomasa A. DeFanti w Uniwersytecie Illinois w Chicago w 1992. Użytkownik jaskini wchodzi do niej w okularach 3D zaopatrzonych w trackery które sprawdzają położenie jego linii wzroku, dzięki czemu obraz może być odpowiednio dopasowany aby stworzyć iluzję głębi obrazu. CAVE w LZWP jest dodatkowo wyposarzony w kontrolery ART Flystick pozwalające na interakcję z wirtualnym środowiskiem.


\section{Narzędzia}
Do produkcji użyliśmy silnika Unreal Engine 5.5 stworzonego przez przedsiębiorstwo Epic Games. Jest to silnik przeznaczony głównie do tworzenia gier 3D. Wybraliśmy go ponieważ zostały stworzone do niego biblioteki pozwalające na proste przenoszenie aplikacji ze środowiska desktopowego na środowisko CAVE.

Użyliśmy 2 bibliotek: nDisplay pozwalającej aplikacji na wyświetlaniu obrazu na klastrze systemu CAVE oraz Dtrack pozwalający na obsługę kontrolerów ART Flystick.

\section{Algorytmy}